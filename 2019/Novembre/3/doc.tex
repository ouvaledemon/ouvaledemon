\documentclass[french,12pt,amstex,a4paper]{article}
\usepackage{xcolor}
\usepackage{listings}
\usepackage{graphicx}
\usepackage{amsfonts}
\usepackage{amsmath}
\usepackage[utf8]{inputenc}
\usepackage[T1]{fontenc}
\usepackage[francais]{babel}
\usepackage{float}
\usepackage{hyperref}

\usepackage{tgpagella}

\begin{document}

\begin{center}
{\bf La délivrance}\\
{\it Le Demon dérangé, le 20 novembre 2019}\\
\end{center}

La délivrance, ce sentiment si familier de légèreté, d'être soulagé d'un fardeau. Ce n'est qu'au cours d'un processus long et difficile qu'on peut espérer l'atteindre. Dès lors qu'il est entamé, ce cheminement vers l'ouverture induit une véritable remise en question des lois de la Nature.\\

Nous passons tous par ces moments de douleurs où on ne peut plus continuer à vivre normalement et où on ne souhaite qu'une chose : se délester. Faire disparaître nos problèmes et recommencer à respirer. Une fois qu'on se retrouve reclus et accroupis et qu'on décide de travailler à s'en débarrasser, impossible de revenir en arrière. Le sentiment de mal-être prend alors des dimensions astronomiques et on ne peut s'empêcher de lutter corps et âme.\\

Et si nous n'étions pas faits pour ça ? Les philosophes s'accordent sur le fait que cet effort justement surhumain de concentration est ce qui nous différencie des êtres divins tels que les princesses.\\

Et puis, si on ne défécait pas, on serait quand même bien dans la merde.

\begin{flushright}
-- DD
\end{flushright}

\end{document}










































