\documentclass[french,amstex,12pt,a5paper]{book}
\usepackage{xcolor}
\usepackage{listings}
\usepackage{graphicx}
\usepackage{amsfonts}
\usepackage{amsmath}
\usepackage[utf8]{inputenc}
\usepackage[T1]{fontenc}
\usepackage[francais]{babel}
\usepackage{float}
\usepackage{hyperref}

\usepackage{tgpagella}

\begin{document}

\title{Les droits : là où notre liberté s'arrête}
\author{Le Demon dérangé}
\date{\vspace{-5ex}}
\maketitle

\tableofcontents

\chapter{Les hommes naissent libres et égaux en droits}


Du fait de notre naissance sur le territoire d'un pays, nous sommes tenus de nous plier aux règles imposées par des inconnus ayant pour seule légitimité l'antériorité de leur contrôle de la région.\\

Comment dire que nous naissons libres alors que si nous cueillons les fruits des arbres à proximité d'autres individus considèrent comme légitime de nous enfermer, et ainsi de nous priver de cette liberté ?\\

Pourquoi devrions-nous subir la loi d'une majorité que nous ne reconnaissons pas ?


\end{document}










































