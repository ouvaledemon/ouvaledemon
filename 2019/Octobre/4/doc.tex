\documentclass[french,12pt,amstex,a4paper]{article}
\usepackage{xcolor}
\usepackage{listings}
\usepackage{graphicx}
\usepackage{amsfonts}
\usepackage{amsmath}
\usepackage[utf8]{inputenc}
\usepackage[T1]{fontenc}
\usepackage[francais]{babel}
\usepackage{float}
\usepackage{hyperref}

\usepackage{tgpagella}

\begin{document}

\begin{center}
{\bf Pomme ou courgette ?}\\
%{\bf Les fruits prennent la place des légumes}
{\it Le Demon dérangé, le 27 octobre 2019}\\
\end{center}

Chaque jour, par bateau, ils bravent des conditions extrêmes et débarquent par milliers sur nos rivages. Quand on se promène dans les rues on ne voit plus qu'eux dans les magasins. Évidemment la diversité est essentielle à l'équilibre, mais est-ce vraiment raisonnable de leur laisser le monopole de nos activités commerçantes de la sorte ?\\

Certes ils sont moins chers, mais ces considérations économiques ne prennent pas en compte le fait qu'il faut faire vivre les économies locales et employer en France pour faire face à la crise du chômage. Et puis, même si on acceptait de se faire envahir, il faut prendre en compte le côté humain : si on ne régule pas, ça devient un véritable trafic. Ils passent des semaines dans un conteneur et arrivent fatigués et abîmés, alors qu'ils ne sont parfois même pas encore mûrs.\\

Nous avons eu la chance de recueillir le témoignage de Bertrand, 51 ans, de Bésiers, très préoccupé par la crise : "{\it Non mais vous savez ces trucs là vous savez ce que j'en pense, à un moment faut arrêter quoi, on nous pompe nos sous avec ces histoires, il faut les renvoyer chez eux}".\\

Des contradicteurs peu scrupuleux affirmeront que je dis tout ça parce que je suis intolérant, voire allergique. Je n'ai qu'une chose à répondre à ces gens là : mon dessert préféré est la tarte tatin, mais une alimentation équilibrée ne peut se passer des légumes, même si les fruits venus d'Amérique du Sud nous envahissent. Il faut manger des légumes locaux et bio.\\

\end{document}










































