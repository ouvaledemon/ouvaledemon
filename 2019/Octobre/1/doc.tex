\documentclass[french,12pt,amstex,a4paper]{article}
\usepackage{xcolor}
\usepackage{listings}
\usepackage{graphicx}
\usepackage{amsfonts}
\usepackage{amsmath}
\usepackage[utf8]{inputenc}
\usepackage[T1]{fontenc}
\usepackage[francais]{babel}
\usepackage{float}
\usepackage{hyperref}

\usepackage{tgpagella}

\begin{document}

\begin{center}
{\bf Villani inaugure la prochaine étape de sa campagne}\\
{\it Le Demon dérangé, le 6 octobre 2019}\\
\end{center}

En vue des élections municipales de 2020, le candidat Cédric Villani a développé une campagne d'un nouveau genre. Afin de véritablement rentrer au contact des parisiens, le célèbre mathématicien a dépassé le stade des \href{http://www.lefigaro.fr/elections/municipales/municipales-a-paris-villani-tire-au-sort-une-de-ses-soutiens-pour-un-petit-dejeuner-20191001}{petits déjeuners, (cf. article du Figaro)}. Depuis le 3 octobre, son équipe a mis en place un protocole où un électeur est tiré au sort tous les jeudis à 15h, et passe une heure en privé avec l'intellectuel.\\

Le témoignage d'Alice, 22 ans, mannequin pour Chanel, première désignée par l'impartiale main du hasard pour partager son expérience avec l'homme-araignée et le plonger ouvertement dans son univers de parisienne lambda, est poignant :\\

"{\it Monsieur Villani est un homme très méticuleux, je pensais que j'allais m'endormir d'ennui mais finalement c'était passionnant, il était très à l'écoute et on a bien pris les temps de visiter tous les points qui me paraissaient importants}"\\

Le reponsable de la communication de la campagne ne nous a pas révélé les conditions précises dans lesquelles se déroulaient ces séances privilégiées, néanmoins, il nous a indiqué que "{\it ces ébats, non pardon ces débats, ont lieu dans des conditions permettant un confort optimal et une réelle libération de la parole}". Des relevés banquaires nous indiqueraient qu'un chambre aurait été réservée par le candidat jeudi dernier dans un Ibis budget au Kremlin-Bicêtre.\\

Ces entretiens, menés au nom de la promesse de campagne de Villani qu'il satisferait les parisiens et les parisiennes, marquent le début d'une nouvelle ère en politique : maintenant on ne nous cache plus qu'on est en train de nous niquer.\\

\end{document}










































