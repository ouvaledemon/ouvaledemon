\documentclass[french,12pt,amstex,a4paper]{article}
\usepackage{xcolor}
\usepackage{listings}
\usepackage{graphicx}
\usepackage{amsfonts}
\usepackage{amsmath}
\usepackage[utf8]{inputenc}
\usepackage[T1]{fontenc}
\usepackage[francais]{babel}
\usepackage{float}
\usepackage{hyperref}

\usepackage{tgpagella}

\begin{document}

\begin{center}
{\bf La biodiversité en Australie}\\
{\it Le Demon dérangé, le 25 octobre 2019}\\
\end{center}

L'autre jour, j'ai lu dans le {\it National Geographic} (cf. \href{https://www.nationalgeographic.fr/animaux/2019/10/des-porcins-observes-en-train-dutiliser-des-outils-pour-la-premiere-fois?fbclid=IwAR2x2m3CVhDZbEIHZMPu5Des0gXgkxhVwv0-csoDljXY6nzQ7QAd-n-L-nw}{l'article}) qu'on avait pour la première fois observé des phacochères en train d'utiliser des outils. C'était juste un morceau d'écorce qu'une maman phacochère nommée Priscilla avait utilisé pour creuser.\\

Ça donne quand même beaucoup de matière à réflexion, notamment sur les doubles standards dans notre société. Dans l'article, l'expert de ces porcins témoigne "ce sont des petits malins". Mais quand mon petit frère a utilisé un bout de bois entre ses dents pour creuser un trou pour y enterrer la crotte de chien qu'il venait de dégotter, mon père n'a pas semblé ravi.\\

Puisqu'on est sur le sujet de la biodiversité, j'aimerais vous parler de mon TPE, vous savez le travail de groupe en première où quelque soit la quantité de travail que vous fournissez vous avez l'impression que les autres en font encore moins, et en plus ça compte pour le bac. Dans mon lycée les matières étaient imposées. Sciences de la vie et Anglais. J'avais des amis dans d'autres lycées qui construisaient des robots en legos ou qui étudiaient les roux (oui vraiment) et la place qu'on devait leur accorder dans la société, mais moi je devais travailler sur les sciences de la vie et l'anglais. Soit. Ça allait être un travail chiant, alors autant pousser le vice à fond, le sujet que mon groupe a choisi (coucou Raphaël et Mattéo) s'intitulait "La biodiversité en Australie". Le plan ? Grand un, le wombat. Grand deux, le renard roux. Grand trois, les aborigènes. Il paraît même qu'on en aurait aperçu utiliser des outils récemment.\\

À force de détruire l'habitat naturel des aborigènes ils sont obligés d'aller vivre dans les villes et de recevoir une éducation "normale". La question se pose alors naturellement : à quand le premier phacochère docteur ? Facile. La question plus intéressante serait : à quand la fin des porcs comme docteurs ?

\end{document}










































