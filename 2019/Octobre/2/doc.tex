\documentclass[french,12pt,amstex,a4paper]{article}
\usepackage{xcolor}
\usepackage{listings}
\usepackage{graphicx}
\usepackage{amsfonts}
\usepackage{amsmath}
\usepackage[utf8]{inputenc}
\usepackage[T1]{fontenc}
\usepackage[francais]{babel}
\usepackage{float}
\usepackage{hyperref}

\usepackage{tgpagella}

\begin{document}

\begin{center}
{\bf Ma boîte}\\
{\it Le Demon dérangé, le 11 octobre 2019}\\
\end{center}

Nous sommes si proches. Quand je parle, j'ai l'impression qu'elle me fait écho, comme une sourde caisse de raisonnance. Elle est si ténue et pourtant j'arrive à me perdre dedans, impossible de se réconcilier avec l'idée de son existence, malgré son omniprésence. Ce sentiment oppressant mais parfois rassurant de limite qu'elle procure fait désormais partie de moi, comme un organe régulateur de réflexion, elle a presque intégré mon système nerveux. Tout compte fait, bien que je puisse me perdre dans son néant, elle reste là pour m'assurer que je n'irai jamais trop loin.\\

C'est une boîte, ma boîte. Six murs innamovibles constituant les limites de mon univers. Elle a un plafond où trop souvent je me heurte, un plancher, étrangement à la fois trop bas et trop proche du plafond, et des murs.

\end{document}










































