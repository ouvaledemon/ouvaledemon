\documentclass[french,amstex,12pt,a5paper]{book}
\usepackage{xcolor}
\usepackage{listings}
\usepackage{graphicx}
\usepackage{amsfonts}
\usepackage{amsmath}
\usepackage[utf8]{inputenc}
\usepackage[T1]{fontenc}
\usepackage[francais]{babel}
\usepackage{float}
\usepackage{hyperref}

\usepackage{tgpagella}

\begin{document}

\title{Dans la poudreuse}
\author{DD - commencée en Janvier 2020}
%\date{\vspace{-5ex}}
\maketitle

\tableofcontents

\chapter{Arrivée dans le village}

Quand E. arriva dans le village, il fut immédiatement pris dans l'engourdissement de la vie brumale de la campagne. Tel un gigantesque ours en hibernation, le hameau, parfaitement inerte à l'extérieur des habitations, n'existait plus que par sa vie intérieure, dans les foyers bouillonnant. À l'image de ses pas lourds ralentis par l'épaisse couche de neige jonchant la route naguère fleurie, la monotonie de l'immensité blanche se fondant avec le ciel d'un gris morose ne fournissait guère assez de matière à son esprit pour l'empêcher de vaguabonder, libre et léger dans ce cadre léthargique et froid. En acceptant ce poste d'huissier dans ce village reclus, il s'était condamné à y vivre, au moins quelques années. Parcourir la rue principale qui joignait l'entrée du village par la route départementale à la mairie, lui sembla durer une éternité. Pourtant, il avait déjà arpenté des yeux la quasi-totalité de la bourgade, organisée en épi le long de l'artère. Quand il atteignit enfin la mairie, il était déjà las de ces vieilles maisons en pierre meulière qui se ressemblaient toutes, des cheminées qui semblaient alimenter le gris appesanti de la brume, et de ce vide omniprésent, chromatique et prégnant.\\

Lorsqu'il aperçut la petite pancarte placardée sur la porte de la mairie indiquant ``Réouverture demain à 14h'', il ne fut point surpris, et le reçut comme une évidence. Il ne connaissait personne dans ce village, et n'avait pour seules indications que le nom de la propriétaire du cabinet où il exercerait, et l'adresse - 16 rue du Centre - du deux pièces où il allait désormais vivre. La recherche de son logement ne constituerait même pas une distraction ; il lui suffirait de chercher des yeux le numéro 16 dans l'unique rue portant un nom, cette artère vidée de sa vie, morne et rappelant une peau d'ours polaire tapissant le sol.\\

Cinq coups résonnèrent à travers le village, depuis le vieux clocher en brique, qui lui firent reprendre conscience de là où il était et de ce qu'il faisait. Il décida d'aller déposer ses affaires chez lui, et de s'y installer confortablement, d'autant plus que la nuit n'allait pas tarder à tomber et que le vent commençait à mordre ses joues. Son appartement était situé au deuxième étage, sous le grenier, d'une de ces vieilles maisons, la façade rendue uniforme par l'abrasion des éléments. Il frappa deux coups secs sur la porte en bois qui semblait minuscule et entendit le bruit d'une chaise en bois frottant contre le sol, puis une voix rauque et fatiguée grognant ``J'arrive''. Une vieille courbée au teint blafard vint entrouvrir la porte. ``C'est vous ?'' lui lança-t-elle. Il acquiesca, sans mot dire, et épousseta la neige qui mouchetait son manteau, avant d'entrer chez Jeanne.\\

Jeanne était une des constantes universelles associées au hameau, elle n'en était pratiquement jamais sorti en soixante ans, depuis qu'elle était née - pourquoi l'aurait-elle fait ? Elle n'avait jamais eu besoin de travailler, ayant hérité de la maison à la mort de sa mère, ni de mener une quelconque bataille juridique : elle était fille unique. Pourtant, la vie l'avait épuisée, érodée, elle était presque absente, évanescente, personne dans le village n'avait souvenir d'avoir interragi avec elle, alors même qu'elle allait au marché chaque semaine. Assombrie par la solitude, aigrie par le froid, elle était proprement devenue un reflet du ciel de la région.\\

Quand E. eut fini de gravir les escaliers, le pas lourd, il put poser ses quelques effets et accrocher son manteau, il était enfin chez lui.





\end{document}









































