\documentclass[french,amstex,12pt,a5paper]{book}
\usepackage{xcolor}
\usepackage{listings}
\usepackage{graphicx}
\usepackage{amsfonts}
\usepackage{amsmath}
\usepackage[utf8]{inputenc}
\usepackage[T1]{fontenc}
\usepackage[francais]{babel}
\usepackage{float}
\usepackage{hyperref}

\usepackage{tgpagella}

\begin{document}

\title{Le titre}
\author{DD}
%\date{\vspace{-5ex}}
\maketitle

\tableofcontents

\chapter{Mordu par l'hiver}

Quand E. arriva dans le village, il fut immédiatement pris dans l'engourdissement de la vie brumale de la campagne. Tel un gigantesque ours en hibernation, le hameau, parfaitement inerte à l'extérieur des habitations, n'existait plus que par sa vie intérieure, dans les foyers bouillonnant. À l'image de ses pas lourds ralentis par l'épaisse couche de neige jonchant la route naguère fleurie, la monotonie de l'immensité blanche se fondant avec le ciel d'un gris morose ne fournissait guère assez de matière à son esprit pour l'empêcher de vaguabonder, libre et léger dans ce cadre léthargique et froid. En acceptant ce poste d'huissier dans ce village reclus, il s'était condamné à y vivre, au moins quelques années. Parcourir la rue principale qui joignait l'entrée du village par la route départementale à la mairie, lui sembla durer une éternité. Pourtant, il avait déjà arpenté des yeux la quasi-totalité de la bourgade, organisée en épi le long de l'artère. Quand il atteignit enfin la mairie, il était déjà las de ces vieilles maisons en pierre ponce qui se ressemblaient toutes, des cheminées qui semblaient alimenter le gris appesanti de la brume, et de ce vide omniprésent, chromatique et prégnant.\\

Lorsqu'il aperçut la petite pancarte placardée sur la porte de la mairie indiquant ``Réouverture demain à 14h'', il ne fut point surpris, et le reçut comme un truisme. Il ne connaissait personne dans ce village, et n'avait pour seules indications que le nom de la propriétaire du cabinet où il exercerait, et l'adresse - 16 rue du Centre - du deux pièces où il allait désormais vivre. La recherche de son logement ne constituerait même pas une distraction ; il lui suffirait de chercher des yeux le numéro 16 dans l'unique rue portant un nom, cette artère vidée de sa vie, morne et rappelant une peau d'ours polaire tapissant le sol.



\end{document}









































