\documentclass[french,12pt,amstex,a4paper]{article}
\usepackage{xcolor}
\usepackage{listings}
\usepackage{graphicx}
\usepackage{amsfonts}
\usepackage{amsmath}
\usepackage[utf8]{inputenc}
\usepackage[T1]{fontenc}
\usepackage[francais]{babel}
\usepackage{float}
\usepackage{hyperref}

\usepackage{tgpagella}

\begin{document}

\begin{center}
{\bf Rhapsodie plantaire}\\
{\it DD, le 10 novembre 2019}\\
\end{center}

L'autre jour y'avait une fille chez moi - si, si, je vous promets - au moment de partir, elle s'arrête, et elle prend un air soucieux. Inquiet, je lui demande ce qui ne va pas. Elle, elle se tord les mains, et elle finit par me dire qu'elle n'a pas pris son pied. Immédiatement, je suis rassuré : je lui assure que je comprends son embarras, comment est-ce qu'elle aurait pu rentrer chez elle sans son pied... Je lui dis donc d'aller le chercher, il doit encore être dans la chambre, et que de toute façon je n'avais pas prévu de l'utiliser avant son retour. Elle, ne bouge pas, et de pied ferme, me dit que je fais l'amour comme un pied. Ne perdant pas pied, je mets les pieds dans le plat et la traite de casse-pieds. Elle me dit que je suis bête comme mes pieds. Je lui coupe l'herbe sous le pied et lui dis que j'en avais gardé sous le pied. C'est le pied ! me dit-elle, furibonde. Je prends mes pieds à mon cou, ça me retire une épine du pied, elle avait du se lever du mauvais pied.

\begin{flushright}
-- DD
\end{flushright}

\end{document}










































