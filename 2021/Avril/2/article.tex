\documentclass[french,12pt,amstex,a4paper]{article}
\usepackage{xcolor}
\usepackage{listings}
\usepackage{graphicx}
\usepackage{amsfonts}
\usepackage{amsmath}
\usepackage[utf8]{inputenc}
\usepackage[T1]{fontenc}
\usepackage[francais]{babel}
\usepackage{float}
\usepackage{hyperref}

\usepackage{tgpagella}

\begin{document}

\begin{center}
{\bf Engagé : ah ça oui !}\\
{\it DD, le 30 avril 2021}\\
\end{center}

Abstract : Étienne, un féministe, ne s'assume pas, donc il en fait des tonnes pour compenser (dans un sens comme dans l'autre).\\

\begin{center}
{\bf Scène 1}\\
{\it Étienne, deux de ses potes mecs, une fille qu'ils ne connaissent pas (Louise) assise sur un banc à un quinzaine de mètres}
\end{center}
\noindent - Mec 1 : Oh putain\\
- Mec 2 : {\it siffle}\\
- Mec 1 : Quel morceau !\\
- Mec 2 : T'as vu ça Étienne ? Tu lui mets combien ?\\
- Étienne : {\it (tout bas, en baissant les yeux)} Je peux pas... c'est pas bien...\\
- Mec 1 : Arrête de bégayer, t'es ému ou quoi ? Elle est bonne mais pas à ce point...\\
{\it rires des deux mecs}\\
\noindent - Mec 1 : Nan mais plus sérieusement, qu'est-ce que t'as ?\\
- Étienne : Rien, c'est vrai qu'elle est jolie.\\
- Mec 2 : Tu lui mets ce qu'elle mérite ?\\
- Étienne : {\it fronce les sourcils}\\
- Mec 1 : Aller fais pas la timide, tu tapes dedans ou pas ?\\
- Étienne : {\it (défait)} Oui, oui, bah oui... {\it (se ressaisit)} En même temps c'est appétissant.\\
- Mec 2 : Ça c'est notre Étienne ! Chiche de l'aborder !\\
- Étienne : {\it (mû par une énergie nouvelle, s'adressant à Louise)} Eh ! Tu connais la différence entre une Ferrari et une érection ?\\
{\it (Louise lève les yeux vers les trois mecs, puis soupire et se replonge sur son portable)}\\
{\it (les deux potes d'Étienne sont surpris et ne savent pas où il veut en venir)}\\
- Étienne : {\it (sûr de son effet)} J'ai pas de Ferrari !\\
{\it (rires gras des mecs, Louise s'en va, ennuyée, mais elle a l'habitude)}\\

\begin{center}
{\bf Scène 2}\\
{\it cette scène se passe le soir après la première, Étienne est encore honteux de lui}\\
{\it Étienne, à table avec ses parents et sa petite sœur, Julie}
\end{center}
\noindent{\it(ils mangent)}\\
- La mère : C'est très gentil Étienne d'avoir insisté pour nous faire à manger ce soir.\\
- Le père : Ouais enfin t'aurais pu laisser ta mère t'aider, je sais que le principal c'est l'intention, mais là c'est trop salé.\\
- Julie : Moi je trouve ça louche ça quand même que tu nous aies pas laissé mettre la table, t'as cassé un truc ? T'as quelque chose à te faire pardonner ?\\
- Étienne : {\it (bas, la bouche pleine, en gromellant)} C'est mon rôle, c'est pas à vous de faire ça.\\
{\it (ils ont terminé de manger, la mère se lève pour commencer à débarrasser)}\\
- Étienne : {\it (hurle à sa mère)} Tu te rassieds !\\
- Le père : Mais qu'est-ce qui te prend ?\\
- Julie : {\it (recrache sa gorgée d'eau)}\\
- Étienne : {\it (se lève, et bouillant s'adresse à sa mère)} Tu perpétues insidieusement l'oppression du patriarcat en t'y conformant à chaque instant. Quel exemple tu montres ? Chacun de tes actes est un boulet de plus accroché au pied de Julie, une brique ajoutée au grand mur de l'exploitation, une morsure à pleines dents dans le muscle de l'émancipation !\\
- La mère : {\it (au père)} Ton fils est devenu fou.\\
- Le père : {\it (à la mère)} Ah non, celui-là c'est le tien...\\
{\it Étienne débarrasse la table, encore fumant, les autres s'en vont en rigolant mais un peu surpris et inquiets.}

\begin{center}
{\bf Scène 3}\\
{\it dans un local de la fac d'Étienne, c'est la réunion des colleurs d'affiches féministes de sa fac}\\
{\it Étienne, un groupe de personnes, principalement des femmes}
\end{center}
\noindent{\it (Étienne se faufile discrètement dans le local, en espérant que personne de l'extérieur ne le voit entrer)}\\
- Fille 1 : Bon, on doit décider de ce qu'on met sur nos affiches, faut marquer les esprits.\\
- Fille 2 : Ouais y'a trop de cas de comportements abusifs sur le campus, des filles qui se plaignent qu'on les aborde en les traitant comme de la viande.\\
- Étienne : Faut qu'ils comprennent ces porcs.\\
- Fille 1 : Donc on part sur quelque chose à propos des comportements abusifs sur le campus ? Ou bien sur la culture du viol ?\\
- Fille 2 : Je propose qu'on s'occupe de ces comportements d'abord, ça met vraiment une mauvaise ambiance je trouve.\\
- Étienne : Je propose une affiche avec "Mangez vos couilles" marqué dessus.\\
- Fille 2 : Haha c'est l'esprit. On va faire quelque chose de plus soft, mais ça fait plaisir que des gens soient révoltés et engagés comme ça.\\
- Fille 1 : On va aussi se répartir en équipes de collages, on n'a qu'à faire deux équipes, la mienne et celle d'Étienne.\\
{\it les gens commencent à se répartir à côté de Étienne et de la fille}\\
{\it Louise entre, voit les gens groupés autour d'Étienne, un contact visuel s'établit entre eux}\\
- Louise : Je crois que je me suis trompé de salle. {\it (elle sort)}\\
- Étienne : {\it (qui était un peu décontenancé, se reprend)} C'est dommage on aurait dû profiter de son erreur pour la convaincre de nous aider !\\
- Fille 1 : Tu perds pas le nord !\\
- Étienne : C'est ça l'engagement, quand ça te prend, ça te lâche plus.\\
{\it Deux personnes entrent subrepticement dans la pièce, à la manière d'Étienne plus tôt, c'est les deux mecs de la première scène}\\

\iffalse
Idées : 
- dans une scène plus tard, à une réunion féministe de sa fac par exemple, la fille de la première scène sera présente et ça sera cringe.
- la chute peut-être que mec 1 et mec 2 sont comme ça aussi, genre tous les mecs sont pareils
\fi

\begin{flushright}
-- DD
\end{flushright}

\end{document}










































