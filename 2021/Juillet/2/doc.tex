\documentclass[french,12pt,amstex,a4paper]{article}
\usepackage{xcolor}
\usepackage{listings}
\usepackage{graphicx}
\usepackage{amsfonts}
\usepackage{amsmath}
\usepackage[utf8]{inputenc}
\usepackage[T1]{fontenc}
\usepackage[francais]{babel}
\usepackage{float}
\usepackage{hyperref}

\usepackage{tgpagella}

\begin{document}

\begin{center}
{\bf Tribune déconnectée}\\
{\it DD, le 7 juillet 2021}\\
\end{center}

TRIBUNE coup de gueule de la classe politique :\\

Si nous écoutions la {\it voix du peuple}, celle de la {\it France du bas}, des citoyens, ou plutôt de nos concitoyens, nous serions vite lassés par leur nouvelle obsession : les politiciens seraient déconnectés de la vie que mènent les français. Il est martelé que le ministre du budget pense que les pains au chocolat coûtent quinze centimes, et que la ministre des transports estime le prix des tickets de métro à quatre euros. C'est aborder la question complètement dans le mauvais sens.

En effet, si les hommes politiques peuvent avoir des difficultés à estimer les prix de quelques objets du quotidien, combien de français seraient capables de chiffrer en équivalent costumes sur mesure la somme nécessaire pour qu'un conseiller municipal s'engage à voter en faveur d'un projet de réamménagement urbain ? Cette déconnexion va dans les deux sens, et pointer du doigt les politiciens est de la part des français une négation de leur responsabilité, et un révélateur de leur égocentrisme sans borne.

Quelle valeur donner à une critique à l'égard d'un politique ayant accepté un pot-de-vin, si celle-ci provient de quelqu'un n'ayant jamais eu plus d'un million sur son compte en banque ni même porté une Rolex ? Quelle proportion de nos compatriotes connaissent réellement les enjeux des choix qui se présentent à nous ? Il y a fort à parier que celle-ci est très faible, et ceci est aisément vérifiable, il suffit de constater qu'il n'y a aucun millionnaire -- c'est-à-dire quelqu'un qui percevrait véritablement l'opportunité que la possession d'une telle somme représente -- qui nous jette la pierre. Ce n'est qu'une illustration parmi une multitude montrant à quel point les français sont loin de mesurer ce qui constitue le quotidien d'un homme politique.

Alors, lorsque nous entendons que la faiblesse de notre ancrage dans la vie réelle nous conduit à prendre des décisions n'allant pas dans l'intérêt du plus grand nombre, nous ne pouvons que conseiller aux détracteurs de se regarder dans le miroir et d'ouvrir leur Bible. Ne se rendent-ils pas coupables d'exactement le même tort lorsqu'ils nous élisent ou destituent sans considération aucune pour les conséquences de leur décision sur nos vies, ni connaissance de la qualité respective des différents appartements de fonction ? Leurs décisions ne sont-elles pas toutes autant faussées par une profonde déconnexion ?

C'est cette déconnexion qui est à la base de l'incompréhension qui touche les français. La tentative d'y pallier en nous catégorisant selon un clivage gauche-droite ne fait qu'ajouter au grotesque de la situation. Chercher à entrevoir les mobiles de nos actions à travers ce prisme est l'équivalent d'essayer de déceler les motifs animant un meurtrier en se fiant uniquement à la couleur de ses yeux. C'est une lecture unidimensionnelle d'un phénomène en réalité plurivalent. En ne pensant qu'à leurs intérêts et à leurs conditions de vie, nos compatriotes se bandent les yeux, et se rendent bien incapables d'appréhender les décisions que nous prenons. Ce point de vue autocentré adopté par le peuple ne lui permet pas de saisir la complexité de la situation, de percevoir que nous sommes motivés par des intérêts divergents des siens. Ainsi, l'animosité résultant de cette incompréhension, de la part des français à l'égard des politiques, est tout à fait malvenue, n'étant que la marque de l'ignorance de l'existence de cet antagonisme.

Nous touchons à présent l'objet de cette tribune, ce qui nous a poussé à prendre la plume dans ce canard. Ce texte s'adresse à la fois à nos lecteurs politiciens, et aux simples citoyens. À toi, élu qui lit cette tribune, entend notre appel à la solidarité. Pour défendre nos intérêts de classe nous devons faire bloc. Arrêtons de nous préoccuper de gens qui ne font que nous mettre des bâtons dans les roues. À vous, français, nous affirmons toute notre détermination à mener à bien cette lutte et à rendre les coups, inspirés par ce que nos camarades en Chine et en Russie ont réussi à accomplir.

À votre absence de confiance en nous brandie en étandard, nous opposons une méfiance symétrique. Puisque notre parole ne semble pas avoir de valeur pour les électeurs, nous n'accordons pas le moindre crédit à la leur.

Les français sont dégoûtés par la classe politique ? Eh bien sachez que ce mépris est mutuel.

\end{document}










































