\documentclass[french,amstex,12pt,a5paper]{book}
\usepackage{xcolor}
\usepackage{listings}
\usepackage{graphicx}
\usepackage{amsfonts}
\usepackage{amsmath}
\usepackage[utf8]{inputenc}
\usepackage[T1]{fontenc}
\usepackage[francais]{babel}
\usepackage{float}
\usepackage{hyperref}

\usepackage{tgpagella}

\begin{document}

\title{Le KK - Trex Bugatti}
\author{DD - commencée en Juillet 2021}
%\date{\vspace{-5ex}}
\maketitle

\begin{center}
 {\bf Préface}\\
\end{center}


Après son roman à succès {\it Le Dessert au Tartare}, Trex Bugatti produit un nouveau chef-d'oeuvre. Cette courte nouvelle est l'histoire d'une vie, rythmée par l'attente et l'angoisse, c'est la narration d'une fuite, périodique et ne cessant qu'avec la vie. C'est également le tableau d'une obsession, d'une idée fixe, d'une envie inassouvissable de se mesurer avec son destin.\\

\begin{center}
 {\bf Le KK}\\
\end{center}

Quand Stéphane Prince eut douze ans, il demanda comme cadeau à son père, qui était proctologue et propriétaire d'un somptueux cabinet, de le laisser assister à un de ses rendez-vous.

``Quand je serai grand

\end{document}










































