\documentclass[french,amstex,12pt,a5paper]{book}
\usepackage{xcolor}
\usepackage{listings}
\usepackage{graphicx}
\usepackage{amsfonts}
\usepackage{amsmath}
\usepackage[utf8]{inputenc}
\usepackage[T1]{fontenc}
\usepackage[francais]{babel}
\usepackage{float}
\usepackage{hyperref}

\usepackage{tgpagella}

\begin{document}

\title{Un service de rêve}
\author{DD - commencée en Juin 2021}
%\date{\vspace{-5ex}}
\maketitle

\tableofcontents

\chapter{Le réveil}

David ouvrit les yeux, il serait volontiers resté un peu plus longtemps sous la couette, mais le bruit dans la cuisine lui indiquait qu'il allait enfin pouvoir partager ce qui lui brûlait la langue depuis déjà quelques heures. Il avait fait un rêve, c'était prémonitoire, il en était persuadé, quelque chose de grand allait bientôt se produire. Il enfila son short et un tshirt, et se dirigea vers la cuisine. Il distingua plus clairement ce qu'il s'y passait, une voix familière -- celle de son collocataire Marc -- échangeait avec une voix qui lui était inconnue, appartenant probablement à un autre trentenaire. Cela contrariait son plan, il ne pourrait pas annoncer d'une traite la nouvelle à son ami. Il se contenta d'arborer une moue goguenarde et un ton légèrement hautain, et alla silencieusement se servir un café. ``David, Bruno, Bruno, David'' énonça Marc, d'un ton fatigué. La discussion fut moins vigoureuse qu'un albatros les ailes prises dans du pétrole.\\

Quand Burno partit, Marc demanda :\\
-- ``Alors ?\\
-- Alors quoi ?\\
-- Bah dis moi ce que tu as envie de dire !\\
-- Comment ça ?\\
-- Arrête ton cinéma je te connais par coeur, quand tu fais cette tête là c'est que tu as quelque chose à annoncer.\\
-- N'importe quoi.\\
-- Tu n'as rien à dire ?\\
-- Je n'ai pas dit ça.\\
-- Ah ! je le savais. Dis alors !\\
-- J'ai fait un rêve cette nuit.\\
-- Ah. Mais ça va ? Ça ne t'a pas empêché de dormir ?\\

\end{document}










































